
\documentclass{article}
%%%%%%%%%%%%%%%%%%%%%%%%%%%%%%%%%%%%%%%%%%%%%%%%%%%%%%%%%%%%%%%%%%%%%%%%%%%%%%%%%%%%%%%%%%%%%%%%%%%%%%%%%%%%%%%%%%%%%%%%%%%%
\usepackage{graphicx}
\usepackage{amsmath}

%TCIDATA{OutputFilter=LATEX.DLL}
%TCIDATA{Created=Fri Apr 02 13:03:44 1999}
%TCIDATA{LastRevised=Thu Sep 21 12:25:18 2000}
%TCIDATA{<META NAME="GraphicsSave" CONTENT="32">}
%TCIDATA{<META NAME="DocumentShell" CONTENT="General\Blank Document">}
%TCIDATA{Language=American English}
%TCIDATA{CSTFile=LaTeX article (bright).cst}

% Macros for Scientific Word 4.0 documents saved with the LaTeX filter.
% Copyright (C) 2002 Mackichan Software, Inc.

\typeout{TCILATEX Macros for Scientific Word 5.0 <13 Feb 2003>.}
\typeout{NOTICE:  This macro file is NOT proprietary and may be 
freely copied and distributed.}
%
\makeatletter

%%%%%%%%%%%%%%%%%%%%%
% pdfTeX related.
\ifx\pdfoutput\relax\let\pdfoutput=\undefined\fi
\newcount\msipdfoutput
\ifx\pdfoutput\undefined
\else
 \ifcase\pdfoutput
 \else 
    \msipdfoutput=1
    \ifx\paperwidth\undefined
    \else
      \ifdim\paperheight=0pt\relax
      \else
        \pdfpageheight\paperheight
      \fi
      \ifdim\paperwidth=0pt\relax
      \else
        \pdfpagewidth\paperwidth
      \fi
    \fi
  \fi  
\fi

%%%%%%%%%%%%%%%%%%%%%
% FMTeXButton
% This is used for putting TeXButtons in the 
% frontmatter of a document. Add a line like
% \QTagDef{FMTeXButton}{101}{} to the filter 
% section of the cst being used. Also add a
% new section containing:
%     [f_101]
%     ALIAS=FMTexButton
%     TAG_TYPE=FIELD
%     TAG_LEADIN=TeX Button:
%
% It also works to put \defs in the preamble after 
% the \input tcilatex
\def\FMTeXButton#1{#1}
%
%%%%%%%%%%%%%%%%%%%%%%
% macros for time
\newcount\@hour\newcount\@minute\chardef\@x10\chardef\@xv60
\def\tcitime{
\def\@time{%
  \@minute\time\@hour\@minute\divide\@hour\@xv
  \ifnum\@hour<\@x 0\fi\the\@hour:%
  \multiply\@hour\@xv\advance\@minute-\@hour
  \ifnum\@minute<\@x 0\fi\the\@minute
  }}%

%%%%%%%%%%%%%%%%%%%%%%
% macro for hyperref and msihyperref
%\@ifundefined{hyperref}{\def\hyperref#1#2#3#4{#2\ref{#4}#3}}{}

\def\x@hyperref#1#2#3{%
   % Turn off various catcodes before reading parameter 4
   \catcode`\~ = 12
   \catcode`\$ = 12
   \catcode`\_ = 12
   \catcode`\# = 12
   \catcode`\& = 12
   \y@hyperref{#1}{#2}{#3}%
}

\def\y@hyperref#1#2#3#4{%
   #2\ref{#4}#3
   \catcode`\~ = 13
   \catcode`\$ = 3
   \catcode`\_ = 8
   \catcode`\# = 6
   \catcode`\& = 4
}

\@ifundefined{hyperref}{\let\hyperref\x@hyperref}{}
\@ifundefined{msihyperref}{\let\msihyperref\x@hyperref}{}




% macro for external program call
\@ifundefined{qExtProgCall}{\def\qExtProgCall#1#2#3#4#5#6{\relax}}{}
%%%%%%%%%%%%%%%%%%%%%%
%
% macros for graphics
%
\def\FILENAME#1{#1}%
%
\def\QCTOpt[#1]#2{%
  \def\QCTOptB{#1}
  \def\QCTOptA{#2}
}
\def\QCTNOpt#1{%
  \def\QCTOptA{#1}
  \let\QCTOptB\empty
}
\def\Qct{%
  \@ifnextchar[{%
    \QCTOpt}{\QCTNOpt}
}
\def\QCBOpt[#1]#2{%
  \def\QCBOptB{#1}%
  \def\QCBOptA{#2}%
}
\def\QCBNOpt#1{%
  \def\QCBOptA{#1}%
  \let\QCBOptB\empty
}
\def\Qcb{%
  \@ifnextchar[{%
    \QCBOpt}{\QCBNOpt}%
}
\def\PrepCapArgs{%
  \ifx\QCBOptA\empty
    \ifx\QCTOptA\empty
      {}%
    \else
      \ifx\QCTOptB\empty
        {\QCTOptA}%
      \else
        [\QCTOptB]{\QCTOptA}%
      \fi
    \fi
  \else
    \ifx\QCBOptA\empty
      {}%
    \else
      \ifx\QCBOptB\empty
        {\QCBOptA}%
      \else
        [\QCBOptB]{\QCBOptA}%
      \fi
    \fi
  \fi
}
\newcount\GRAPHICSTYPE
%\GRAPHICSTYPE 0 is for TurboTeX
%\GRAPHICSTYPE 1 is for DVIWindo (PostScript)
%%%(removed)%\GRAPHICSTYPE 2 is for psfig (PostScript)
\GRAPHICSTYPE=\z@
\def\GRAPHICSPS#1{%
 \ifcase\GRAPHICSTYPE%\GRAPHICSTYPE=0
   \special{ps: #1}%
 \or%\GRAPHICSTYPE=1
   \special{language "PS", include "#1"}%
%%%\or%\GRAPHICSTYPE=2
%%%  #1%
 \fi
}%
%
\def\GRAPHICSHP#1{\special{include #1}}%
%
% \graffile{ body }                                  %#1
%          { contentswidth (scalar)  }               %#2
%          { contentsheight (scalar) }               %#3
%          { vertical shift when in-line (scalar) }  %#4

\def\graffile#1#2#3#4{%
%%% \ifnum\GRAPHICSTYPE=\tw@
%%%  %Following if using psfig
%%%  \@ifundefined{psfig}{\input psfig.tex}{}%
%%%  \psfig{file=#1, height=#3, width=#2}%
%%% \else
  %Following for all others
  % JCS - added BOXTHEFRAME, see below
    \bgroup
	   \@inlabelfalse
       \leavevmode
       \@ifundefined{bbl@deactivate}{\def~{\string~}}{\activesoff}%
        \raise -#4 \BOXTHEFRAME{%
           \hbox to #2{\raise #3\hbox to #2{\null #1\hfil}}}%
    \egroup
}%
%
% A box for drafts
\def\draftbox#1#2#3#4{%
 \leavevmode\raise -#4 \hbox{%
  \frame{\rlap{\protect\tiny #1}\hbox to #2%
   {\vrule height#3 width\z@ depth\z@\hfil}%
  }%
 }%
}%
%
\newcount\@msidraft
\@msidraft=\z@
\let\nographics=\@msidraft
\newif\ifwasdraft
\wasdraftfalse

%  \GRAPHIC{ body }                                  %#1
%          { draft name }                            %#2
%          { contentswidth (scalar)  }               %#3
%          { contentsheight (scalar) }               %#4
%          { vertical shift when in-line (scalar) }  %#5
\def\GRAPHIC#1#2#3#4#5{%
   \ifnum\@msidraft=\@ne\draftbox{#2}{#3}{#4}{#5}%
   \else\graffile{#1}{#3}{#4}{#5}%
   \fi
}
%
\def\addtoLaTeXparams#1{%
    \edef\LaTeXparams{\LaTeXparams #1}}%
%
% JCS -  added a switch BoxFrame that can 
% be set by including X in the frame params.
% If set a box is drawn around the frame.

\newif\ifBoxFrame \BoxFramefalse
\newif\ifOverFrame \OverFramefalse
\newif\ifUnderFrame \UnderFramefalse

\def\BOXTHEFRAME#1{%
   \hbox{%
      \ifBoxFrame
         \frame{#1}%
      \else
         {#1}%
      \fi
   }%
}


\def\doFRAMEparams#1{\BoxFramefalse\OverFramefalse\UnderFramefalse\readFRAMEparams#1\end}%
\def\readFRAMEparams#1{%
 \ifx#1\end%
  \let\next=\relax
  \else
  \ifx#1i\dispkind=\z@\fi
  \ifx#1d\dispkind=\@ne\fi
  \ifx#1f\dispkind=\tw@\fi
  \ifx#1t\addtoLaTeXparams{t}\fi
  \ifx#1b\addtoLaTeXparams{b}\fi
  \ifx#1p\addtoLaTeXparams{p}\fi
  \ifx#1h\addtoLaTeXparams{h}\fi
  \ifx#1X\BoxFrametrue\fi
  \ifx#1O\OverFrametrue\fi
  \ifx#1U\UnderFrametrue\fi
  \ifx#1w
    \ifnum\@msidraft=1\wasdrafttrue\else\wasdraftfalse\fi
    \@msidraft=\@ne
  \fi
  \let\next=\readFRAMEparams
  \fi
 \next
 }%
%
%Macro for In-line graphics object
%   \IFRAME{ contentswidth (scalar)  }               %#1
%          { contentsheight (scalar) }               %#2
%          { vertical shift when in-line (scalar) }  %#3
%          { draft name }                            %#4
%          { body }                                  %#5
%          { caption}                                %#6


\def\IFRAME#1#2#3#4#5#6{%
      \bgroup
      \let\QCTOptA\empty
      \let\QCTOptB\empty
      \let\QCBOptA\empty
      \let\QCBOptB\empty
      #6%
      \parindent=0pt
      \leftskip=0pt
      \rightskip=0pt
      \setbox0=\hbox{\QCBOptA}%
      \@tempdima=#1\relax
      \ifOverFrame
          % Do this later
          \typeout{This is not implemented yet}%
          \show\HELP
      \else
         \ifdim\wd0>\@tempdima
            \advance\@tempdima by \@tempdima
            \ifdim\wd0 >\@tempdima
               \setbox1 =\vbox{%
                  \unskip\hbox to \@tempdima{\hfill\GRAPHIC{#5}{#4}{#1}{#2}{#3}\hfill}%
                  \unskip\hbox to \@tempdima{\parbox[b]{\@tempdima}{\QCBOptA}}%
               }%
               \wd1=\@tempdima
            \else
               \textwidth=\wd0
               \setbox1 =\vbox{%
                 \noindent\hbox to \wd0{\hfill\GRAPHIC{#5}{#4}{#1}{#2}{#3}\hfill}\\%
                 \noindent\hbox{\QCBOptA}%
               }%
               \wd1=\wd0
            \fi
         \else
            \ifdim\wd0>0pt
              \hsize=\@tempdima
              \setbox1=\vbox{%
                \unskip\GRAPHIC{#5}{#4}{#1}{#2}{0pt}%
                \break
                \unskip\hbox to \@tempdima{\hfill \QCBOptA\hfill}%
              }%
              \wd1=\@tempdima
           \else
              \hsize=\@tempdima
              \setbox1=\vbox{%
                \unskip\GRAPHIC{#5}{#4}{#1}{#2}{0pt}%
              }%
              \wd1=\@tempdima
           \fi
         \fi
         \@tempdimb=\ht1
         %\advance\@tempdimb by \dp1
         \advance\@tempdimb by -#2
         \advance\@tempdimb by #3
         \leavevmode
         \raise -\@tempdimb \hbox{\box1}%
      \fi
      \egroup%
}%
%
%Macro for Display graphics object
%   \DFRAME{ contentswidth (scalar)  }               %#1
%          { contentsheight (scalar) }               %#2
%          { draft label }                           %#3
%          { name }                                  %#4
%          { caption}                                %#5
\def\DFRAME#1#2#3#4#5{%
  \vspace\topsep
  \hfil\break
  \bgroup
     \leftskip\@flushglue
	 \rightskip\@flushglue
	 \parindent\z@
	 \parfillskip\z@skip
     \let\QCTOptA\empty
     \let\QCTOptB\empty
     \let\QCBOptA\empty
     \let\QCBOptB\empty
	 \vbox\bgroup
        \ifOverFrame 
           #5\QCTOptA\par
        \fi
        \GRAPHIC{#4}{#3}{#1}{#2}{\z@}%
        \ifUnderFrame 
           \break#5\QCBOptA
        \fi
	 \egroup
  \egroup
  \vspace\topsep
  \break
}%
%
%Macro for Floating graphic object
%   \FFRAME{ framedata f|i tbph x F|T }              %#1
%          { contentswidth (scalar)  }               %#2
%          { contentsheight (scalar) }               %#3
%          { caption }                               %#4
%          { label }                                 %#5
%          { draft name }                            %#6
%          { body }                                  %#7
\def\FFRAME#1#2#3#4#5#6#7{%
 %If float.sty loaded and float option is 'h', change to 'H'  (gp) 1998/09/05
  \@ifundefined{floatstyle}
    {%floatstyle undefined (and float.sty not present), no change
     \begin{figure}[#1]%
    }
    {%floatstyle DEFINED
	 \ifx#1h%Only the h parameter, change to H
      \begin{figure}[H]%
	 \else
      \begin{figure}[#1]%
	 \fi
	}
  \let\QCTOptA\empty
  \let\QCTOptB\empty
  \let\QCBOptA\empty
  \let\QCBOptB\empty
  \ifOverFrame
    #4
    \ifx\QCTOptA\empty
    \else
      \ifx\QCTOptB\empty
        \caption{\QCTOptA}%
      \else
        \caption[\QCTOptB]{\QCTOptA}%
      \fi
    \fi
    \ifUnderFrame\else
      \label{#5}%
    \fi
  \else
    \UnderFrametrue%
  \fi
  \begin{center}\GRAPHIC{#7}{#6}{#2}{#3}{\z@}\end{center}%
  \ifUnderFrame
    #4
    \ifx\QCBOptA\empty
      \caption{}%
    \else
      \ifx\QCBOptB\empty
        \caption{\QCBOptA}%
      \else
        \caption[\QCBOptB]{\QCBOptA}%
      \fi
    \fi
    \label{#5}%
  \fi
  \end{figure}%
 }%
%
%
%    \FRAME{ framedata f|i tbph x F|T }              %#1
%          { contentswidth (scalar)  }               %#2
%          { contentsheight (scalar) }               %#3
%          { vertical shift when in-line (scalar) }  %#4
%          { caption }                               %#5
%          { label }                                 %#6
%          { name }                                  %#7
%          { body }                                  %#8
%
%    framedata is a string which can contain the following
%    characters: idftbphxFT
%    Their meaning is as follows:
%             i, d or f : in-line, display, or floating
%             t,b,p,h   : LaTeX floating placement options
%             x         : fit contents box to contents
%             F or T    : Figure or Table. 
%                         Later this can expand
%                         to a more general float class.
%
%
\newcount\dispkind%

\def\makeactives{
  \catcode`\"=\active
  \catcode`\;=\active
  \catcode`\:=\active
  \catcode`\'=\active
  \catcode`\~=\active
}
\bgroup
   \makeactives
   \gdef\activesoff{%
      \def"{\string"}%
      \def;{\string;}%
      \def:{\string:}%
      \def'{\string'}%
      \def~{\string~}%
      %\bbl@deactivate{"}%
      %\bbl@deactivate{;}%
      %\bbl@deactivate{:}%
      %\bbl@deactivate{'}%
    }
\egroup

\def\FRAME#1#2#3#4#5#6#7#8{%
 \bgroup
 \ifnum\@msidraft=\@ne
   \wasdrafttrue
 \else
   \wasdraftfalse%
 \fi
 \def\LaTeXparams{}%
 \dispkind=\z@
 \def\LaTeXparams{}%
 \doFRAMEparams{#1}%
 \ifnum\dispkind=\z@\IFRAME{#2}{#3}{#4}{#7}{#8}{#5}\else
  \ifnum\dispkind=\@ne\DFRAME{#2}{#3}{#7}{#8}{#5}\else
   \ifnum\dispkind=\tw@
    \edef\@tempa{\noexpand\FFRAME{\LaTeXparams}}%
    \@tempa{#2}{#3}{#5}{#6}{#7}{#8}%
    \fi
   \fi
  \fi
  \ifwasdraft\@msidraft=1\else\@msidraft=0\fi{}%
  \egroup
 }%
%
% This macro added to let SW gobble a parameter that
% should not be passed on and expanded. 

\def\TEXUX#1{"texux"}

%
% Macros for text attributes:
%
\def\BF#1{{\bf {#1}}}%
\def\NEG#1{\leavevmode\hbox{\rlap{\thinspace/}{$#1$}}}%
%
%%%%%%%%%%%%%%%%%%%%%%%%%%%%%%%%%%%%%%%%%%%%%%%%%%%%%%%%%%%%%%%%%%%%%%%%
%
%
% macros for user - defined functions
\def\limfunc#1{\mathop{\rm #1}}%
\def\func#1{\mathop{\rm #1}\nolimits}%
% macro for unit names
\def\unit#1{\mathord{\thinspace\rm #1}}%

%
% miscellaneous 
\long\def\QQQ#1#2{%
     \long\expandafter\def\csname#1\endcsname{#2}}%
\@ifundefined{QTP}{\def\QTP#1{}}{}
\@ifundefined{QEXCLUDE}{\def\QEXCLUDE#1{}}{}
\@ifundefined{Qlb}{\def\Qlb#1{#1}}{}
\@ifundefined{Qlt}{\def\Qlt#1{#1}}{}
\def\QWE{}%
\long\def\QQA#1#2{}%
\def\QTR#1#2{{\csname#1\endcsname {#2}}}%
\long\def\TeXButton#1#2{#2}%
\long\def\QSubDoc#1#2{#2}%
\def\EXPAND#1[#2]#3{}%
\def\NOEXPAND#1[#2]#3{}%
\def\PROTECTED{}%
\def\LaTeXparent#1{}%
\def\ChildStyles#1{}%
\def\ChildDefaults#1{}%
\def\QTagDef#1#2#3{}%

% Constructs added with Scientific Notebook
\@ifundefined{correctchoice}{\def\correctchoice{\relax}}{}
\@ifundefined{HTML}{\def\HTML#1{\relax}}{}
\@ifundefined{TCIIcon}{\def\TCIIcon#1#2#3#4{\relax}}{}
\if@compatibility
  \typeout{Not defining UNICODE  U or CustomNote commands for LaTeX 2.09.}
\else
  \providecommand{\UNICODE}[2][]{\protect\rule{.1in}{.1in}}
  \providecommand{\U}[1]{\protect\rule{.1in}{.1in}}
  \providecommand{\CustomNote}[3][]{\marginpar{#3}}
\fi

\@ifundefined{lambdabar}{
      \def\lambdabar{\errmessage{You have used the lambdabar symbol. 
                      This is available for typesetting only in RevTeX styles.}}
   }{}

%
% Macros for style editor docs
\@ifundefined{StyleEditBeginDoc}{\def\StyleEditBeginDoc{\relax}}{}
%
% Macros for footnotes
\def\QQfnmark#1{\footnotemark}
\def\QQfntext#1#2{\addtocounter{footnote}{#1}\footnotetext{#2}}
%
% Macros for indexing.
%
\@ifundefined{TCIMAKEINDEX}{}{\makeindex}%
%
% Attempts to avoid problems with other styles
\@ifundefined{abstract}{%
 \def\abstract{%
  \if@twocolumn
   \section*{Abstract (Not appropriate in this style!)}%
   \else \small 
   \begin{center}{\bf Abstract\vspace{-.5em}\vspace{\z@}}\end{center}%
   \quotation 
   \fi
  }%
 }{%
 }%
\@ifundefined{endabstract}{\def\endabstract
  {\if@twocolumn\else\endquotation\fi}}{}%
\@ifundefined{maketitle}{\def\maketitle#1{}}{}%
\@ifundefined{affiliation}{\def\affiliation#1{}}{}%
\@ifundefined{proof}{\def\proof{\noindent{\bfseries Proof. }}}{}%
\@ifundefined{endproof}{\def\endproof{\mbox{\ \rule{.1in}{.1in}}}}{}%
\@ifundefined{newfield}{\def\newfield#1#2{}}{}%
\@ifundefined{chapter}{\def\chapter#1{\par(Chapter head:)#1\par }%
 \newcount\c@chapter}{}%
\@ifundefined{part}{\def\part#1{\par(Part head:)#1\par }}{}%
\@ifundefined{section}{\def\section#1{\par(Section head:)#1\par }}{}%
\@ifundefined{subsection}{\def\subsection#1%
 {\par(Subsection head:)#1\par }}{}%
\@ifundefined{subsubsection}{\def\subsubsection#1%
 {\par(Subsubsection head:)#1\par }}{}%
\@ifundefined{paragraph}{\def\paragraph#1%
 {\par(Subsubsubsection head:)#1\par }}{}%
\@ifundefined{subparagraph}{\def\subparagraph#1%
 {\par(Subsubsubsubsection head:)#1\par }}{}%
%%%%%%%%%%%%%%%%%%%%%%%%%%%%%%%%%%%%%%%%%%%%%%%%%%%%%%%%%%%%%%%%%%%%%%%%
% These symbols are not recognized by LaTeX
\@ifundefined{therefore}{\def\therefore{}}{}%
\@ifundefined{backepsilon}{\def\backepsilon{}}{}%
\@ifundefined{yen}{\def\yen{\hbox{\rm\rlap=Y}}}{}%
\@ifundefined{registered}{%
   \def\registered{\relax\ifmmode{}\r@gistered
                    \else$\m@th\r@gistered$\fi}%
 \def\r@gistered{^{\ooalign
  {\hfil\raise.07ex\hbox{$\scriptstyle\rm\text{R}$}\hfil\crcr
  \mathhexbox20D}}}}{}%
\@ifundefined{Eth}{\def\Eth{}}{}%
\@ifundefined{eth}{\def\eth{}}{}%
\@ifundefined{Thorn}{\def\Thorn{}}{}%
\@ifundefined{thorn}{\def\thorn{}}{}%
% A macro to allow any symbol that requires math to appear in text
\def\TEXTsymbol#1{\mbox{$#1$}}%
\@ifundefined{degree}{\def\degree{{}^{\circ}}}{}%
%
% macros for T3TeX files
\newdimen\theight
\@ifundefined{Column}{\def\Column{%
 \vadjust{\setbox\z@=\hbox{\scriptsize\quad\quad tcol}%
  \theight=\ht\z@\advance\theight by \dp\z@\advance\theight by \lineskip
  \kern -\theight \vbox to \theight{%
   \rightline{\rlap{\box\z@}}%
   \vss
   }%
  }%
 }}{}%
%
\@ifundefined{qed}{\def\qed{%
 \ifhmode\unskip\nobreak\fi\ifmmode\ifinner\else\hskip5\p@\fi\fi
 \hbox{\hskip5\p@\vrule width4\p@ height6\p@ depth1.5\p@\hskip\p@}%
 }}{}%
%
\@ifundefined{cents}{\def\cents{\hbox{\rm\rlap c/}}}{}%
\@ifundefined{tciLaplace}{\def\tciLaplace{\ensuremath{\mathcal{L}}}}{}%
\@ifundefined{tciFourier}{\def\tciFourier{\ensuremath{\mathcal{F}}}}{}%
\@ifundefined{textcurrency}{\def\textcurrency{\hbox{\rm\rlap xo}}}{}%
\@ifundefined{texteuro}{\def\texteuro{\hbox{\rm\rlap C=}}}{}%
\@ifundefined{euro}{\def\euro{\hbox{\rm\rlap C=}}}{}%
\@ifundefined{textfranc}{\def\textfranc{\hbox{\rm\rlap-F}}}{}%
\@ifundefined{textlira}{\def\textlira{\hbox{\rm\rlap L=}}}{}%
\@ifundefined{textpeseta}{\def\textpeseta{\hbox{\rm P\negthinspace s}}}{}%
%
\@ifundefined{miss}{\def\miss{\hbox{\vrule height2\p@ width 2\p@ depth\z@}}}{}%
%
\@ifundefined{vvert}{\def\vvert{\Vert}}{}%  %always translated to \left| or \right|
%
\@ifundefined{tcol}{\def\tcol#1{{\baselineskip=6\p@ \vcenter{#1}} \Column}}{}%
%
\@ifundefined{dB}{\def\dB{\hbox{{}}}}{}%        %dummy entry in column 
\@ifundefined{mB}{\def\mB#1{\hbox{$#1$}}}{}%   %column entry
\@ifundefined{nB}{\def\nB#1{\hbox{#1}}}{}%     %column entry (not math)
%
\@ifundefined{note}{\def\note{$^{\dag}}}{}%
%
\def\newfmtname{LaTeX2e}
% No longer load latexsym.  This is now handled by SWP, which uses amsfonts if necessary
%
\ifx\fmtname\newfmtname
  \DeclareOldFontCommand{\rm}{\normalfont\rmfamily}{\mathrm}
  \DeclareOldFontCommand{\sf}{\normalfont\sffamily}{\mathsf}
  \DeclareOldFontCommand{\tt}{\normalfont\ttfamily}{\mathtt}
  \DeclareOldFontCommand{\bf}{\normalfont\bfseries}{\mathbf}
  \DeclareOldFontCommand{\it}{\normalfont\itshape}{\mathit}
  \DeclareOldFontCommand{\sl}{\normalfont\slshape}{\@nomath\sl}
  \DeclareOldFontCommand{\sc}{\normalfont\scshape}{\@nomath\sc}
\fi

%
% Greek bold macros
% Redefine all of the math symbols 
% which might be bolded	 - there are 
% probably others to add to this list

\def\alpha{{\Greekmath 010B}}%
\def\beta{{\Greekmath 010C}}%
\def\gamma{{\Greekmath 010D}}%
\def\delta{{\Greekmath 010E}}%
\def\epsilon{{\Greekmath 010F}}%
\def\zeta{{\Greekmath 0110}}%
\def\eta{{\Greekmath 0111}}%
\def\theta{{\Greekmath 0112}}%
\def\iota{{\Greekmath 0113}}%
\def\kappa{{\Greekmath 0114}}%
\def\lambda{{\Greekmath 0115}}%
\def\mu{{\Greekmath 0116}}%
\def\nu{{\Greekmath 0117}}%
\def\xi{{\Greekmath 0118}}%
\def\pi{{\Greekmath 0119}}%
\def\rho{{\Greekmath 011A}}%
\def\sigma{{\Greekmath 011B}}%
\def\tau{{\Greekmath 011C}}%
\def\upsilon{{\Greekmath 011D}}%
\def\phi{{\Greekmath 011E}}%
\def\chi{{\Greekmath 011F}}%
\def\psi{{\Greekmath 0120}}%
\def\omega{{\Greekmath 0121}}%
\def\varepsilon{{\Greekmath 0122}}%
\def\vartheta{{\Greekmath 0123}}%
\def\varpi{{\Greekmath 0124}}%
\def\varrho{{\Greekmath 0125}}%
\def\varsigma{{\Greekmath 0126}}%
\def\varphi{{\Greekmath 0127}}%

\def\nabla{{\Greekmath 0272}}
\def\FindBoldGroup{%
   {\setbox0=\hbox{$\mathbf{x\global\edef\theboldgroup{\the\mathgroup}}$}}%
}

\def\Greekmath#1#2#3#4{%
    \if@compatibility
        \ifnum\mathgroup=\symbold
           \mathchoice{\mbox{\boldmath$\displaystyle\mathchar"#1#2#3#4$}}%
                      {\mbox{\boldmath$\textstyle\mathchar"#1#2#3#4$}}%
                      {\mbox{\boldmath$\scriptstyle\mathchar"#1#2#3#4$}}%
                      {\mbox{\boldmath$\scriptscriptstyle\mathchar"#1#2#3#4$}}%
        \else
           \mathchar"#1#2#3#4% 
        \fi 
    \else 
        \FindBoldGroup
        \ifnum\mathgroup=\theboldgroup % For 2e
           \mathchoice{\mbox{\boldmath$\displaystyle\mathchar"#1#2#3#4$}}%
                      {\mbox{\boldmath$\textstyle\mathchar"#1#2#3#4$}}%
                      {\mbox{\boldmath$\scriptstyle\mathchar"#1#2#3#4$}}%
                      {\mbox{\boldmath$\scriptscriptstyle\mathchar"#1#2#3#4$}}%
        \else
           \mathchar"#1#2#3#4% 
        \fi     	    
	  \fi}

\newif\ifGreekBold  \GreekBoldfalse
\let\SAVEPBF=\pbf
\def\pbf{\GreekBoldtrue\SAVEPBF}%
%

\@ifundefined{theorem}{\newtheorem{theorem}{Theorem}}{}
\@ifundefined{lemma}{\newtheorem{lemma}[theorem]{Lemma}}{}
\@ifundefined{corollary}{\newtheorem{corollary}[theorem]{Corollary}}{}
\@ifundefined{conjecture}{\newtheorem{conjecture}[theorem]{Conjecture}}{}
\@ifundefined{proposition}{\newtheorem{proposition}[theorem]{Proposition}}{}
\@ifundefined{axiom}{\newtheorem{axiom}{Axiom}}{}
\@ifundefined{remark}{\newtheorem{remark}{Remark}}{}
\@ifundefined{example}{\newtheorem{example}{Example}}{}
\@ifundefined{exercise}{\newtheorem{exercise}{Exercise}}{}
\@ifundefined{definition}{\newtheorem{definition}{Definition}}{}


\@ifundefined{mathletters}{%
  %\def\theequation{\arabic{equation}}
  \newcounter{equationnumber}  
  \def\mathletters{%
     \addtocounter{equation}{1}
     \edef\@currentlabel{\theequation}%
     \setcounter{equationnumber}{\c@equation}
     \setcounter{equation}{0}%
     \edef\theequation{\@currentlabel\noexpand\alph{equation}}%
  }
  \def\endmathletters{%
     \setcounter{equation}{\value{equationnumber}}%
  }
}{}

%Logos
\@ifundefined{BibTeX}{%
    \def\BibTeX{{\rm B\kern-.05em{\sc i\kern-.025em b}\kern-.08em
                 T\kern-.1667em\lower.7ex\hbox{E}\kern-.125emX}}}{}%
\@ifundefined{AmS}%
    {\def\AmS{{\protect\usefont{OMS}{cmsy}{m}{n}%
                A\kern-.1667em\lower.5ex\hbox{M}\kern-.125emS}}}{}%
\@ifundefined{AmSTeX}{\def\AmSTeX{\protect\AmS-\protect\TeX\@}}{}%
%

% This macro is a fix to eqnarray
\def\@@eqncr{\let\@tempa\relax
    \ifcase\@eqcnt \def\@tempa{& & &}\or \def\@tempa{& &}%
      \else \def\@tempa{&}\fi
     \@tempa
     \if@eqnsw
        \iftag@
           \@taggnum
        \else
           \@eqnnum\stepcounter{equation}%
        \fi
     \fi
     \global\tag@false
     \global\@eqnswtrue
     \global\@eqcnt\z@\cr}


\def\TCItag{\@ifnextchar*{\@TCItagstar}{\@TCItag}}
\def\@TCItag#1{%
    \global\tag@true
    \global\def\@taggnum{(#1)}}
\def\@TCItagstar*#1{%
    \global\tag@true
    \global\def\@taggnum{#1}}
%
%%%%%%%%%%%%%%%%%%%%%%%%%%%%%%%%%%%%%%%%%%%%%%%%%%%%%%%%%%%%%%%%%%%%%
%
\def\QATOP#1#2{{#1 \atop #2}}%
\def\QTATOP#1#2{{\textstyle {#1 \atop #2}}}%
\def\QDATOP#1#2{{\displaystyle {#1 \atop #2}}}%
\def\QABOVE#1#2#3{{#2 \above#1 #3}}%
\def\QTABOVE#1#2#3{{\textstyle {#2 \above#1 #3}}}%
\def\QDABOVE#1#2#3{{\displaystyle {#2 \above#1 #3}}}%
\def\QOVERD#1#2#3#4{{#3 \overwithdelims#1#2 #4}}%
\def\QTOVERD#1#2#3#4{{\textstyle {#3 \overwithdelims#1#2 #4}}}%
\def\QDOVERD#1#2#3#4{{\displaystyle {#3 \overwithdelims#1#2 #4}}}%
\def\QATOPD#1#2#3#4{{#3 \atopwithdelims#1#2 #4}}%
\def\QTATOPD#1#2#3#4{{\textstyle {#3 \atopwithdelims#1#2 #4}}}%
\def\QDATOPD#1#2#3#4{{\displaystyle {#3 \atopwithdelims#1#2 #4}}}%
\def\QABOVED#1#2#3#4#5{{#4 \abovewithdelims#1#2#3 #5}}%
\def\QTABOVED#1#2#3#4#5{{\textstyle 
   {#4 \abovewithdelims#1#2#3 #5}}}%
\def\QDABOVED#1#2#3#4#5{{\displaystyle 
   {#4 \abovewithdelims#1#2#3 #5}}}%
%
% Macros for text size operators:
%
\def\tint{\mathop{\textstyle \int}}%
\def\tiint{\mathop{\textstyle \iint }}%
\def\tiiint{\mathop{\textstyle \iiint }}%
\def\tiiiint{\mathop{\textstyle \iiiint }}%
\def\tidotsint{\mathop{\textstyle \idotsint }}%
\def\toint{\mathop{\textstyle \oint}}%
\def\tsum{\mathop{\textstyle \sum }}%
\def\tprod{\mathop{\textstyle \prod }}%
\def\tbigcap{\mathop{\textstyle \bigcap }}%
\def\tbigwedge{\mathop{\textstyle \bigwedge }}%
\def\tbigoplus{\mathop{\textstyle \bigoplus }}%
\def\tbigodot{\mathop{\textstyle \bigodot }}%
\def\tbigsqcup{\mathop{\textstyle \bigsqcup }}%
\def\tcoprod{\mathop{\textstyle \coprod }}%
\def\tbigcup{\mathop{\textstyle \bigcup }}%
\def\tbigvee{\mathop{\textstyle \bigvee }}%
\def\tbigotimes{\mathop{\textstyle \bigotimes }}%
\def\tbiguplus{\mathop{\textstyle \biguplus }}%
%
%
%Macros for display size operators:
%
\def\dint{\mathop{\displaystyle \int}}%
\def\diint{\mathop{\displaystyle \iint}}%
\def\diiint{\mathop{\displaystyle \iiint}}%
\def\diiiint{\mathop{\displaystyle \iiiint }}%
\def\didotsint{\mathop{\displaystyle \idotsint }}%
\def\doint{\mathop{\displaystyle \oint}}%
\def\dsum{\mathop{\displaystyle \sum }}%
\def\dprod{\mathop{\displaystyle \prod }}%
\def\dbigcap{\mathop{\displaystyle \bigcap }}%
\def\dbigwedge{\mathop{\displaystyle \bigwedge }}%
\def\dbigoplus{\mathop{\displaystyle \bigoplus }}%
\def\dbigodot{\mathop{\displaystyle \bigodot }}%
\def\dbigsqcup{\mathop{\displaystyle \bigsqcup }}%
\def\dcoprod{\mathop{\displaystyle \coprod }}%
\def\dbigcup{\mathop{\displaystyle \bigcup }}%
\def\dbigvee{\mathop{\displaystyle \bigvee }}%
\def\dbigotimes{\mathop{\displaystyle \bigotimes }}%
\def\dbiguplus{\mathop{\displaystyle \biguplus }}%

\if@compatibility\else
  % Always load amsmath in LaTeX2e mode
  \RequirePackage{amsmath}
\fi

\def\ExitTCILatex{\makeatother\endinput}

\bgroup
\ifx\ds@amstex\relax
   \message{amstex already loaded}\aftergroup\ExitTCILatex
\else
   \@ifpackageloaded{amsmath}%
      {\if@compatibility\message{amsmath already loaded}\fi\aftergroup\ExitTCILatex}
      {}
   \@ifpackageloaded{amstex}%
      {\if@compatibility\message{amstex already loaded}\fi\aftergroup\ExitTCILatex}
      {}
   \@ifpackageloaded{amsgen}%
      {\if@compatibility\message{amsgen already loaded}\fi\aftergroup\ExitTCILatex}
      {}
\fi
\egroup

%Exit if any of the AMS macros are already loaded.
%This is always the case for LaTeX2e mode.


%%%%%%%%%%%%%%%%%%%%%%%%%%%%%%%%%%%%%%%%%%%%%%%%%%%%%%%%%%%%%%%%%%%%%%%%%%
% NOTE: The rest of this file is read only if in LaTeX 2.09 compatibility
% mode. This section is used to define AMS-like constructs in the
% event they have not been defined.
%%%%%%%%%%%%%%%%%%%%%%%%%%%%%%%%%%%%%%%%%%%%%%%%%%%%%%%%%%%%%%%%%%%%%%%%%%
\typeout{TCILATEX defining AMS-like constructs in LaTeX 2.09 COMPATIBILITY MODE}
%%%%%%%%%%%%%%%%%%%%%%%%%%%%%%%%%%%%%%%%%%%%%%%%%%%%%%%%%%%%%%%%%%%%%%%%
%  Macros to define some AMS LaTeX constructs when 
%  AMS LaTeX has not been loaded
% 
% These macros are copied from the AMS-TeX package for doing
% multiple integrals.
%
\let\DOTSI\relax
\def\RIfM@{\relax\ifmmode}%
\def\FN@{\futurelet\next}%
\newcount\intno@
\def\iint{\DOTSI\intno@\tw@\FN@\ints@}%
\def\iiint{\DOTSI\intno@\thr@@\FN@\ints@}%
\def\iiiint{\DOTSI\intno@4 \FN@\ints@}%
\def\idotsint{\DOTSI\intno@\z@\FN@\ints@}%
\def\ints@{\findlimits@\ints@@}%
\newif\iflimtoken@
\newif\iflimits@
\def\findlimits@{\limtoken@true\ifx\next\limits\limits@true
 \else\ifx\next\nolimits\limits@false\else
 \limtoken@false\ifx\ilimits@\nolimits\limits@false\else
 \ifinner\limits@false\else\limits@true\fi\fi\fi\fi}%
\def\multint@{\int\ifnum\intno@=\z@\intdots@                          %1
 \else\intkern@\fi                                                    %2
 \ifnum\intno@>\tw@\int\intkern@\fi                                   %3
 \ifnum\intno@>\thr@@\int\intkern@\fi                                 %4
 \int}%                                                               %5
\def\multintlimits@{\intop\ifnum\intno@=\z@\intdots@\else\intkern@\fi
 \ifnum\intno@>\tw@\intop\intkern@\fi
 \ifnum\intno@>\thr@@\intop\intkern@\fi\intop}%
\def\intic@{%
    \mathchoice{\hskip.5em}{\hskip.4em}{\hskip.4em}{\hskip.4em}}%
\def\negintic@{\mathchoice
 {\hskip-.5em}{\hskip-.4em}{\hskip-.4em}{\hskip-.4em}}%
\def\ints@@{\iflimtoken@                                              %1
 \def\ints@@@{\iflimits@\negintic@
   \mathop{\intic@\multintlimits@}\limits                             %2
  \else\multint@\nolimits\fi                                          %3
  \eat@}%                                                             %4
 \else                                                                %5
 \def\ints@@@{\iflimits@\negintic@
  \mathop{\intic@\multintlimits@}\limits\else
  \multint@\nolimits\fi}\fi\ints@@@}%
\def\intkern@{\mathchoice{\!\!\!}{\!\!}{\!\!}{\!\!}}%
\def\plaincdots@{\mathinner{\cdotp\cdotp\cdotp}}%
\def\intdots@{\mathchoice{\plaincdots@}%
 {{\cdotp}\mkern1.5mu{\cdotp}\mkern1.5mu{\cdotp}}%
 {{\cdotp}\mkern1mu{\cdotp}\mkern1mu{\cdotp}}%
 {{\cdotp}\mkern1mu{\cdotp}\mkern1mu{\cdotp}}}%
%
%
%  These macros are for doing the AMS \text{} construct
%
\def\RIfM@{\relax\protect\ifmmode}
\def\text{\RIfM@\expandafter\text@\else\expandafter\mbox\fi}
\let\nfss@text\text
\def\text@#1{\mathchoice
   {\textdef@\displaystyle\f@size{#1}}%
   {\textdef@\textstyle\tf@size{\firstchoice@false #1}}%
   {\textdef@\textstyle\sf@size{\firstchoice@false #1}}%
   {\textdef@\textstyle \ssf@size{\firstchoice@false #1}}%
   \glb@settings}

\def\textdef@#1#2#3{\hbox{{%
                    \everymath{#1}%
                    \let\f@size#2\selectfont
                    #3}}}
\newif\iffirstchoice@
\firstchoice@true
%
%These are the AMS constructs for multiline limits.
%
\def\Let@{\relax\iffalse{\fi\let\\=\cr\iffalse}\fi}%
\def\vspace@{\def\vspace##1{\crcr\noalign{\vskip##1\relax}}}%
\def\multilimits@{\bgroup\vspace@\Let@
 \baselineskip\fontdimen10 \scriptfont\tw@
 \advance\baselineskip\fontdimen12 \scriptfont\tw@
 \lineskip\thr@@\fontdimen8 \scriptfont\thr@@
 \lineskiplimit\lineskip
 \vbox\bgroup\ialign\bgroup\hfil$\m@th\scriptstyle{##}$\hfil\crcr}%
\def\Sb{_\multilimits@}%
\def\endSb{\crcr\egroup\egroup\egroup}%
\def\Sp{^\multilimits@}%
\let\endSp\endSb
%
%
%These are AMS constructs for horizontal arrows
%
\newdimen\ex@
\ex@.2326ex
\def\rightarrowfill@#1{$#1\m@th\mathord-\mkern-6mu\cleaders
 \hbox{$#1\mkern-2mu\mathord-\mkern-2mu$}\hfill
 \mkern-6mu\mathord\rightarrow$}%
\def\leftarrowfill@#1{$#1\m@th\mathord\leftarrow\mkern-6mu\cleaders
 \hbox{$#1\mkern-2mu\mathord-\mkern-2mu$}\hfill\mkern-6mu\mathord-$}%
\def\leftrightarrowfill@#1{$#1\m@th\mathord\leftarrow
\mkern-6mu\cleaders
 \hbox{$#1\mkern-2mu\mathord-\mkern-2mu$}\hfill
 \mkern-6mu\mathord\rightarrow$}%
\def\overrightarrow{\mathpalette\overrightarrow@}%
\def\overrightarrow@#1#2{\vbox{\ialign{##\crcr\rightarrowfill@#1\crcr
 \noalign{\kern-\ex@\nointerlineskip}$\m@th\hfil#1#2\hfil$\crcr}}}%
\let\overarrow\overrightarrow
\def\overleftarrow{\mathpalette\overleftarrow@}%
\def\overleftarrow@#1#2{\vbox{\ialign{##\crcr\leftarrowfill@#1\crcr
 \noalign{\kern-\ex@\nointerlineskip}$\m@th\hfil#1#2\hfil$\crcr}}}%
\def\overleftrightarrow{\mathpalette\overleftrightarrow@}%
\def\overleftrightarrow@#1#2{\vbox{\ialign{##\crcr
   \leftrightarrowfill@#1\crcr
 \noalign{\kern-\ex@\nointerlineskip}$\m@th\hfil#1#2\hfil$\crcr}}}%
\def\underrightarrow{\mathpalette\underrightarrow@}%
\def\underrightarrow@#1#2{\vtop{\ialign{##\crcr$\m@th\hfil#1#2\hfil
  $\crcr\noalign{\nointerlineskip}\rightarrowfill@#1\crcr}}}%
\let\underarrow\underrightarrow
\def\underleftarrow{\mathpalette\underleftarrow@}%
\def\underleftarrow@#1#2{\vtop{\ialign{##\crcr$\m@th\hfil#1#2\hfil
  $\crcr\noalign{\nointerlineskip}\leftarrowfill@#1\crcr}}}%
\def\underleftrightarrow{\mathpalette\underleftrightarrow@}%
\def\underleftrightarrow@#1#2{\vtop{\ialign{##\crcr$\m@th
  \hfil#1#2\hfil$\crcr
 \noalign{\nointerlineskip}\leftrightarrowfill@#1\crcr}}}%
%%%%%%%%%%%%%%%%%%%%%

\def\qopnamewl@#1{\mathop{\operator@font#1}\nlimits@}
\let\nlimits@\displaylimits
\def\setboxz@h{\setbox\z@\hbox}


\def\varlim@#1#2{\mathop{\vtop{\ialign{##\crcr
 \hfil$#1\m@th\operator@font lim$\hfil\crcr
 \noalign{\nointerlineskip}#2#1\crcr
 \noalign{\nointerlineskip\kern-\ex@}\crcr}}}}

 \def\rightarrowfill@#1{\m@th\setboxz@h{$#1-$}\ht\z@\z@
  $#1\copy\z@\mkern-6mu\cleaders
  \hbox{$#1\mkern-2mu\box\z@\mkern-2mu$}\hfill
  \mkern-6mu\mathord\rightarrow$}
\def\leftarrowfill@#1{\m@th\setboxz@h{$#1-$}\ht\z@\z@
  $#1\mathord\leftarrow\mkern-6mu\cleaders
  \hbox{$#1\mkern-2mu\copy\z@\mkern-2mu$}\hfill
  \mkern-6mu\box\z@$}


\def\projlim{\qopnamewl@{proj\,lim}}
\def\injlim{\qopnamewl@{inj\,lim}}
\def\varinjlim{\mathpalette\varlim@\rightarrowfill@}
\def\varprojlim{\mathpalette\varlim@\leftarrowfill@}
\def\varliminf{\mathpalette\varliminf@{}}
\def\varliminf@#1{\mathop{\underline{\vrule\@depth.2\ex@\@width\z@
   \hbox{$#1\m@th\operator@font lim$}}}}
\def\varlimsup{\mathpalette\varlimsup@{}}
\def\varlimsup@#1{\mathop{\overline
  {\hbox{$#1\m@th\operator@font lim$}}}}

%
%Companion to stackrel
\def\stackunder#1#2{\mathrel{\mathop{#2}\limits_{#1}}}%
%
%
% These are AMS environments that will be defined to
% be verbatims if amstex has not actually been 
% loaded
%
%
\begingroup \catcode `|=0 \catcode `[= 1
\catcode`]=2 \catcode `\{=12 \catcode `\}=12
\catcode`\\=12 
|gdef|@alignverbatim#1\end{align}[#1|end[align]]
|gdef|@salignverbatim#1\end{align*}[#1|end[align*]]

|gdef|@alignatverbatim#1\end{alignat}[#1|end[alignat]]
|gdef|@salignatverbatim#1\end{alignat*}[#1|end[alignat*]]

|gdef|@xalignatverbatim#1\end{xalignat}[#1|end[xalignat]]
|gdef|@sxalignatverbatim#1\end{xalignat*}[#1|end[xalignat*]]

|gdef|@gatherverbatim#1\end{gather}[#1|end[gather]]
|gdef|@sgatherverbatim#1\end{gather*}[#1|end[gather*]]

|gdef|@gatherverbatim#1\end{gather}[#1|end[gather]]
|gdef|@sgatherverbatim#1\end{gather*}[#1|end[gather*]]


|gdef|@multilineverbatim#1\end{multiline}[#1|end[multiline]]
|gdef|@smultilineverbatim#1\end{multiline*}[#1|end[multiline*]]

|gdef|@arraxverbatim#1\end{arrax}[#1|end[arrax]]
|gdef|@sarraxverbatim#1\end{arrax*}[#1|end[arrax*]]

|gdef|@tabulaxverbatim#1\end{tabulax}[#1|end[tabulax]]
|gdef|@stabulaxverbatim#1\end{tabulax*}[#1|end[tabulax*]]


|endgroup
  

  
\def\align{\@verbatim \frenchspacing\@vobeyspaces \@alignverbatim
You are using the "align" environment in a style in which it is not defined.}
\let\endalign=\endtrivlist
 
\@namedef{align*}{\@verbatim\@salignverbatim
You are using the "align*" environment in a style in which it is not defined.}
\expandafter\let\csname endalign*\endcsname =\endtrivlist




\def\alignat{\@verbatim \frenchspacing\@vobeyspaces \@alignatverbatim
You are using the "alignat" environment in a style in which it is not defined.}
\let\endalignat=\endtrivlist
 
\@namedef{alignat*}{\@verbatim\@salignatverbatim
You are using the "alignat*" environment in a style in which it is not defined.}
\expandafter\let\csname endalignat*\endcsname =\endtrivlist




\def\xalignat{\@verbatim \frenchspacing\@vobeyspaces \@xalignatverbatim
You are using the "xalignat" environment in a style in which it is not defined.}
\let\endxalignat=\endtrivlist
 
\@namedef{xalignat*}{\@verbatim\@sxalignatverbatim
You are using the "xalignat*" environment in a style in which it is not defined.}
\expandafter\let\csname endxalignat*\endcsname =\endtrivlist




\def\gather{\@verbatim \frenchspacing\@vobeyspaces \@gatherverbatim
You are using the "gather" environment in a style in which it is not defined.}
\let\endgather=\endtrivlist
 
\@namedef{gather*}{\@verbatim\@sgatherverbatim
You are using the "gather*" environment in a style in which it is not defined.}
\expandafter\let\csname endgather*\endcsname =\endtrivlist


\def\multiline{\@verbatim \frenchspacing\@vobeyspaces \@multilineverbatim
You are using the "multiline" environment in a style in which it is not defined.}
\let\endmultiline=\endtrivlist
 
\@namedef{multiline*}{\@verbatim\@smultilineverbatim
You are using the "multiline*" environment in a style in which it is not defined.}
\expandafter\let\csname endmultiline*\endcsname =\endtrivlist


\def\arrax{\@verbatim \frenchspacing\@vobeyspaces \@arraxverbatim
You are using a type of "array" construct that is only allowed in AmS-LaTeX.}
\let\endarrax=\endtrivlist

\def\tabulax{\@verbatim \frenchspacing\@vobeyspaces \@tabulaxverbatim
You are using a type of "tabular" construct that is only allowed in AmS-LaTeX.}
\let\endtabulax=\endtrivlist

 
\@namedef{arrax*}{\@verbatim\@sarraxverbatim
You are using a type of "array*" construct that is only allowed in AmS-LaTeX.}
\expandafter\let\csname endarrax*\endcsname =\endtrivlist

\@namedef{tabulax*}{\@verbatim\@stabulaxverbatim
You are using a type of "tabular*" construct that is only allowed in AmS-LaTeX.}
\expandafter\let\csname endtabulax*\endcsname =\endtrivlist

% macro to simulate ams tag construct


% This macro is a fix to the equation environment
 \def\endequation{%
     \ifmmode\ifinner % FLEQN hack
      \iftag@
        \addtocounter{equation}{-1} % undo the increment made in the begin part
        $\hfil
           \displaywidth\linewidth\@taggnum\egroup \endtrivlist
        \global\tag@false
        \global\@ignoretrue   
      \else
        $\hfil
           \displaywidth\linewidth\@eqnnum\egroup \endtrivlist
        \global\tag@false
        \global\@ignoretrue 
      \fi
     \else   
      \iftag@
        \addtocounter{equation}{-1} % undo the increment made in the begin part
        \eqno \hbox{\@taggnum}
        \global\tag@false%
        $$\global\@ignoretrue
      \else
        \eqno \hbox{\@eqnnum}% $$ BRACE MATCHING HACK
        $$\global\@ignoretrue
      \fi
     \fi\fi
 } 

 \newif\iftag@ \tag@false
 
 \def\TCItag{\@ifnextchar*{\@TCItagstar}{\@TCItag}}
 \def\@TCItag#1{%
     \global\tag@true
     \global\def\@taggnum{(#1)}}
 \def\@TCItagstar*#1{%
     \global\tag@true
     \global\def\@taggnum{#1}}

  \@ifundefined{tag}{
     \def\tag{\@ifnextchar*{\@tagstar}{\@tag}}
     \def\@tag#1{%
         \global\tag@true
         \global\def\@taggnum{(#1)}}
     \def\@tagstar*#1{%
         \global\tag@true
         \global\def\@taggnum{#1}}
  }{}

\def\tfrac#1#2{{\textstyle {#1 \over #2}}}%
\def\dfrac#1#2{{\displaystyle {#1 \over #2}}}%
\def\binom#1#2{{#1 \choose #2}}%
\def\tbinom#1#2{{\textstyle {#1 \choose #2}}}%
\def\dbinom#1#2{{\displaystyle {#1 \choose #2}}}%

% Do not add anything to the end of this file.  
% The last section of the file is loaded only if 
% amstex has not been.
\makeatother
\endinput

\setlength{\oddsidemargin}{0in}
\setlength{\textwidth}{6.5in}
\setlength{\topmargin}{-.5in}
\setlength{\textheight}{9in}

\begin{document}


\noindent {\LARGE How to Estimate Structural Vector Autoregressions\ using
Matlab}

{\Large by Robert Vigfusson\footnote{{\Large My email address is
r-vigfusson@nwu.edu. Updates to these programs will be made available on my
webpage htttp://pubweb.nwu.edu/\symbol{126}rjv541. }}}

{\large May 10, 1999}

\bigskip

I have written a set of Matlab M-files to estimate structural vector
autoregressions (VAR). This note describes how to use these files.

Each of the following sections is divided into two subsections: Theory and
Matlab Code . The theory is a condensed description of the main concepts of
structural VARs. The descriptions borrow heavily from Christiano, Eichenbaum
and Evans (1998) Handbook chapter and Hamilton (1994)'s time series text.%
\footnote{%
One therefore should see these two sources for a more complete treatment. Of
course, comments on how to improve the exposition here are welcome.} The
Matlab Code subsection should give the user enough information to use the
procedures for his own work. \ At the end of the paper, I give examples of
how to estimate specific structural VARs using the functions described here.

The sections are the following

\begin{itemize}
\item  \qquad\ Set up and Estimation of Reduced Form VAR

\item  \qquad Fundamental Shocks and The Structural VAR

\item  \qquad Impulse Responses

\item  \qquad Constructing Confidence Intervals for Impulse Responses

\item  \qquad Variance Decomposition

\item  \qquad Examples
\end{itemize}

\textbf{Software Requirements} These programs require Matlab 5.
Non-recursive estimation requires the optimization toolbox. The current code
is written for optimization toolbox version 1.5. 

\section{Set up and Estimation of Reduced Form VAR}

\subsection{Theory}

The basic building block is a vector time series $Z_{t}$ with $nvars$
elements (variables). The vector $Z_{t}$ evolves over time according to:.

\begin{equation*}
Z_{t}=B_{0}+B_{1}Z_{t-1}+B_{2}Z_{t-2}+...+B_{q}Z_{t-q}+u_{t}
\end{equation*}
where $Eu_{t}u_{t}^{\prime }=V.$ The vector $u_{t}$ is assumed to be
uncorrelated with past values of $Z_{t}.$

Estimation of the VAR's coefficients $\left\{ B_{j}\right\} _{j=0}^{q}$ can
be done equation by equation by OLS. The matrix $V$ can be estimated from
the sample residuals $\left( 1/T\right) \sum \hat{u}_{t}\hat{u}_{t}^{\prime
} $.

\subsection{Matlab Code}

The first step is to prepare the data. In a spreadsheet or statistical
program, enter the data such that each column is a variable and each row is
an observation. There should be no missing observations. Save this file to
an ascii text file. In order that Matlab can read the file, each row should
have the same number of entries. Also, do not include variable names with
the data -- just numbers. (I'll assume the file is named \emph{datrep.txt.) }%
Next start Matlab and enter the following two commands. (Of course, this can
be done in a script m-file.) 
\begin{equation*}
\text{\texttt{load datrep.txt}}
\end{equation*}
\begin{equation*}
\text{\texttt{data = datrep;}}
\end{equation*}

The next step is to define a variable for the number of lags in the VAR. If
the number of lags $q$ is 4 then we define

\begin{equation*}
\text{nlags \ }\mathtt{=4;}
\end{equation*}

By default, the program includes a constant in the VAR. If the VAR\ doesn't
have a constant then define a variable $hascon$ and set it equal to zero.

\begin{equation*}
\text{\texttt{hascon}}=0
\end{equation*}
If $hascon$ is set equal to one, then a constant is included in the VAR.

The next line estimates the VAR. The syntax is

\begin{equation*}
\text{\texttt{\lbrack betaz,sigma,residuals]}}=\text{\texttt{%
estimatevar(data,nlags,hascon);}}
\end{equation*}

The procedure \texttt{estimatevar} returns three variables. The first $\
betaz$ is the coefficients of the VAR. The second $sigma$ is the variance
covariance matrix. The third \ $residuals$ is a matrix of the residuals from
the VAR. The $residuals$ are organized the same way as the data matrix is
organized. Each column is a variable and each row is an observation.

The matrix $betaz$ has $nvars$ rows and $nlags$ times $nvars$+$hascon$
columns. When a constant is included, the first column is the constant.
Columns two to $nvars+1$ are the estimated coefficients for the variables $%
Z_{t-1}$. The next $nvars$ columns would be the estimated \ coefficients for
the $Z_{t-2}.$Without a constant, the first nvars columns are the estimated
coefficients for the variables $Z_{t-1}.$

For a 2 variable (bivariate) VAR with 3 lags, $betaz$ would equal 
\begin{equation*}
\text{betaz}=\left[ 
\begin{array}{cccc}
B_{0} & B_{1} & B_{2} & B_{3}
\end{array}
\right]
\end{equation*}
where $B_{j}$ is defined above.

The procedure is written with a default to include a constant. Therefore one
does not have to include the $hascon$ variable if you want to estimate the
VAR\ with a constant. 
\begin{equation*}
\text{\texttt{\lbrack betaz,sigma]=estimatevar(data,nlags);}}
\end{equation*}

\subparagraph{General Notes on Using Matlab}

The position and value of a variable and not its name is what is important
when using a Matlab function. For example, $nlags$, the number of lags, was
used in the \texttt{estimatevar} command. 
\begin{equation*}
\text{\texttt{\lbrack betaz,sigma]=estimatevar(data,nlags);}}
\end{equation*}
The value and not the name, however, is what is important. If $nlags$ were
equal to four then 
\begin{equation*}
\mathtt{\lbrack betaz,sigma]=estimatevar(data,4);}
\end{equation*}
would produce exactly the same output. I define the variable $nlags$ as I
use it in subsequent procedures.

Another thing to keep in mind is that one can use the command\texttt{\ help}
to find out more about a function in Matlab. For example \texttt{help min }$%
\ $\ will give instructions on how to use Matlab's \texttt{min }command. The
commands described here also have help comments. For example 
\begin{equation*}
\text{\texttt{help estimatevar}}
\end{equation*}
returns
\begin{verbatim}
ESTIMATEVAR Estimates a Vector Autoregression.
 [Beta,Sigma,UZ]=ESTIMATEVAR(DATA,nlags,hasconstant);
 Finds coefficients estimates BETA, variance covariance matrix SIGMA, and residuals UZ
 that are the results of estimate by OLS a VAR on the matrix DATA.
 The matrix DATA has each row being an observations
 from all the variables. NLAGS is the number of lags.
 HASCONSTANT is equal to one for a VAR with a constant and equal to zero otherwise.
\end{verbatim}

\section{Fundamental Shocks and The Structural VAR}

\subsection{Theory}

The next step is to calculate the structural matrix $A_{0}$. The values of $%
u_{t}$ are not the standard structural shocks. Instead we assume that the
relationship between the VAR disturbances $u_{t}$ and the fundamental
economic shocks $\varepsilon _{t}$ is given by

\begin{equation*}
A_{o}u_{t}=\varepsilon _{t}
\end{equation*}

\begin{eqnarray*}
A_{o}Z_{t}
&=&A_{o}B_{1}Z_{t-1}+A_{o}B_{2}Z_{t-2}+...+A_{o}B_{q}Z_{t-q}+A_{o}u_{t} \\
&=&A_{o}B_{1}Z_{t-1}+A_{o}B_{2}Z_{t-2}+...+A_{o}B_{q}Z_{t-q}+\varepsilon _{t}
\end{eqnarray*}

We need to know the value of $A_{o}$ in order to calculate impulse responses
and variance decompositions. To find a unique $A_{o}$\ however requires
further assumptions. Many matrices might solve the equation

\begin{equation*}
A_{o}^{-1}\left( A_{o}^{\prime }\right) ^{-1}=V
\end{equation*}

In general one can choose between two different approaches. The simpler and
more common is the recursiveness assumption. One implements this assumption
by defining the matrix $A_{o}^{-1}$ to be the lower Cholesky decomposition
of $V.$

The other approach is to not to assume recursiveness but rather assume that
enough entries are zero or otherwise constrained that only a unique $A_{o}$
solves the above equation and satisfies these further assumptions. However
the uniqueness of this $A_{o}$ can be difficult to establish.

\subsection{Matlab Code}

\subsubsection{$A_{0}$ with the Recursive Assumption}

Finding the value of $A_{0}$ is much easier with the recursiveness
assumption than with the non-recursiveness assumptions. Recursiveness is
straightforward. We just define $A_{0}$ to be

\begin{equation*}
\text{\texttt{a0}}=\text{\texttt{inv(chol(sigma)');}}
\end{equation*}

The transpose operator is necessary since Matlab's \texttt{chol }command
creates an upper rather than lower triangular matrix. The inverse is used in
order to define $A_{0}$ rather than $A_{0}^{-1}.$

\subsubsection{$A_{0}$ without the Recursive Assumption}

Non-recursiveness is not as straightforward. \ The matrix $A_{0}$ is found
as the matrix that maximize the likelihood function (Hamilton p. 332)

\begin{equation*}
L\left( A_{o},V\right) =-\frac{T}{2}\log \left( 2\pi \right) +\frac{T}{2}%
\log \left| A_{o}\right| ^{2}-\frac{T}{2}\text{trace}(A_{o}VA_{o}^{\prime })
\end{equation*}

The solution is found by using the constrained minimization routine \texttt{%
constr}\footnote{%
The computer solves the maximization problem by minimizing $-1\ast
L(A_{0},V).$}. I use \texttt{constr} rather than \texttt{fminu} in order to
impose the constraint that the diagonal elements of $A_{o}$ are non-negative%
\footnote{%
A user who understands the constr function could add additional constraints
by modifying the matlab function that calculates the likelihood and the
constraints.}.

To estimate the matrix $A_{o}$, the user must write two functions. The first
function forms the desired matrix $A_{o}$ from a vector of parameters $x$.
The second function generates a vector $x$ of starting values that satisfy
the constraints placed on $A_{0}.$ This vector $x$ will be used as starting
values in the numerical optimization procedure.

An example of the first function that creates the $A_{o}$ matrix of Sims and
Zha is below.
\begin{verbatim}
function [azero]=mkmatrix(x);
x=x';
azero=[x(1:7); 0 x(8:9) 0 -x(8) 0 -x(8); x(10:12)zeros(1,4); x(13) . . .
zeros(1,2) x(14:17);x(18) zeros(1,3) x(19:21) ;x(22) zeros(1,4) x(23:24);x(25) ...
zeros(1,5) x(26)];
\end{verbatim}

This function creates a matrix that looks like

\begin{equation*}
A= 
\begin{array}{ccccccc}
x(1) & x(2) & x(3) & x(4) & x(5) & x(6) & x(7) \\ 
0 & x(8) & x(9) & 0 & -x(8) & 0 & -x(8) \\ 
x(10) & x(11) & x(12) & 0 & 0 & 0 & 0 \\ 
x(13) & 0 & 0 & x(14) & x(15) & x(16) & x(17) \\ 
x(18) & 0 & 0 & 0 & x(19) & x(20) & x(21) \\ 
x(22) & 0 & 0 & 0 & 0 & x(23) & x(24) \\ 
x(25) & 0 & 0 & 0 & 0 & 0 & x(26)
\end{array}
\end{equation*}

The second function creates a vector $x$ that will be used as the initial
guess to the maximization routine. An example for the Sims Zha matrix is
below:

\texttt{function avec = mkstart}

\%Generate Random Starting Values

\texttt{avec=2*randn(26,1); \%random starting values}

\texttt{\%making the diagonal elements non-negative}

\texttt{avec([1 8 12 14 19 23 26]) = abs(avec([1 8 12 14 19 23 26]));}

Numerical Maximization can be difficult. Common problems include failing to
converge or converging at a local rather than a global maximum. To overcome
these problems, I recommend to estimate the matrix $A_{0}$ many times using
random starting values and then choose the estimate with the greatest
likelihood value. The procedure \texttt{estnonreca} implements this
strategy. Given the variance covariance matrix $sigma$ and the user supplied
procedures \texttt{mkmatrix} and \texttt{mkstart} supplied above, \texttt{%
estnonreca} estimates the $A_{0}$ matrix $numtries$ times and returns the
best estimate out of the set\textbf{. \ }

\begin{equation*}
\mathtt{a0best=estnonreca(sigma,numtries,\prime mkmatrix\prime ,\prime
mkstart\prime )}
\end{equation*}

\section{Impulse Responses}

\subsection{Theory}

Having found the $A_{0}$ matrix, the next step is to calculate the impulse
responses to a fundamental shock. The basic idea is the following. Suppose
that the VAR has the following form.

\begin{equation*}
Z_{t}=B_{1}Z_{t-1}+B_{2}Z_{t-2}+...+B_{q}Z_{t-q}+A_{o}^{-1}\varepsilon _{t}
\end{equation*}

We suppose that the $j-th$ fundamental errors take on the value one while
the other fundamental errors are set equal to zero. The impulse response is
how the variables in the VAR respond to this shock. The impulse response in
the first period is 
\begin{equation*}
\Gamma _{j}(1)=A_{o}^{-1}e_{j}
\end{equation*}
where the j-th element of $e_{j}$ is equal to one and all other elements are
zero. The vector $\Gamma _{j}(1)$ has length $nvars$ with the same ordering
as $Z_{t}.$ The second period impulse response is

\begin{equation*}
\Gamma _{j}(2)=B_{1}A_{o}^{-1}e_{j}
\end{equation*}

The third is

\begin{equation*}
\Gamma _{j}(3)=B_{1}B_{1}A_{o}^{-1}e_{j}+B_{2}A_{o}^{-1}e_{j}
\end{equation*}

Note that if the matrix $A_{o}$ were the identity matrix, then the impulse
response functions for all shocks would be the moving average representation
of the VAR.

\subsection{Matlab Code}

The Matlab syntax for calculating the impulse response function is the
following. Define a variable that determines which error term will receive
the shock. Define another variable for how many periods to calculate the
impulse responses.

\begin{eqnarray*}
\text{errshk} &=&\text{4;} \\
\text{nstep} &=&\text{15;}
\end{eqnarray*}

The next step calls the procedure that actually calculate the impulse
responses. The procedure returns a $nstep$ by $nvars$ matrix with each
column corresponding to a variable in the original VAR and each row by the
period.

\begin{equation*}
\text{impzmat = mkimprep(betaz,a0,nlags,errshk,nstep);}
\end{equation*}

After discussing confidence intervals, I will describe how to plot these
impulse responses.

\section{Constructing Confidence Intervals for Impulse Responses}

The procedures use simulation to calculate the confidence intervals around
the impulse response functions. \ These confidence intervals are pointwise
confidence intervals.

The procedure \texttt{mkimpci }constructs the confidence interval with the
following steps. The first step is to generate the artificial data under the
null hypothesis that the estimated model correctly represents the data. The
artificial data is created by using the estimated coefficients for $beta$
along with error terms. These error terms can be generated using either of
two possible methods: Bootstrap or Monte Carlo. For the Bootstrap method,
the error terms are generated by sampling randomly with replacement from the
residuals of the original vector autoregression. (Note that the $nvars$
residuals for a single time period are chosen together.) For the Monte Carlo
method, the error terms are generated by multiplying random vectors drawn
from a N(0,I) distribution by $A_{o}^{-1}$.

Having generated the artificial data, the next step is to estimate a VAR on
this data. Using the estimated variance covariance matrix, estimate a new $%
A_{0}$ matrix. The impulse response is calculated using these new estimates$%
. $ These impulse responses are then stored in memory.\footnote{%
If memory constraints were a problem, the impulse responses could be stored
to disk instead. This would require\ a modification of the \texttt{mkimpci}
procedure.}

The above steps are repeated many times to generate a large sample of
impulse responses. The procedure then calculates the confidence intervals
using two different methods. The first uses $\alpha $ and $1-\alpha $
percentile values of the sample for the lower and upper bounds. The second
relies on the normal approximation and calculates the confidence intervals
as two times the square root of the sample variance around the estimated
impulse response function.

\subsection{Matlab code}

Before calling the mkimpci procedure three variables need to be defined. The
first variable $ndraws$ defines how many simulations to use in constructing
the confidence intervals. The second variable $nobs$ defines how many
observations each simulated time series should have. Typically this is taken
to be the number of observations in the actual data. (To eliminate the
effects of initial conditions, the procedure generates $2nobs$ observations
and then uses only the last $nobs$ of data.) The third variable $pctg$ gives
the value of $\alpha $ to be used in constructing the confidence intervals.

\begin{eqnarray*}
\text{pctg} &=&0.05; \\
\text{ndraws} &=&100 \\
\text{nobs} &=&120;
\end{eqnarray*}

\subsubsection{Recursive}

The confidence intervals are constructed by calling the procedure \texttt{%
mkimpci}. The first five input variables are the same as the \texttt{mkimprep%
}'s inputs.\ These are followed by the three defined above. The procedure
returns the percentile lower and upper bounds first, followed by the upper
and lower standard deviation bounds, and finally the sample variance of the
impulse responses. Each matrix is organized the same way as the impulse
responses (variables columns, periods rows).

\begin{equation*}
\text{\lbrack cilmc,ciumc,cilvarm,ciuvarm,varm]}=\text{%
mkimpci(betaz,a0,nlags,errshk,nstep,ndraws,nobs,pctg);}
\end{equation*}

This first call will calculate the confidence intervals based upon the Monte
Carlo simulated data. If the matrix of residuals from the \texttt{estimatevar%
} command is included as an input then the function calculates the
confidence intervals based upon the Boot Strapped simulated data.

\begin{equation*}
\text{\lbrack
cilbt,ciubt,cilvarb,ciuvarb,varb]=mkimpci(betaz,a0,nlags,errshk,nstep,ndraws,nobs,pctg,residuals);%
}
\end{equation*}

\subsubsection{Non-recursive}

The non-recursive system is identical except that the four additional
parameters from the \texttt{estnonreca} procedure are also included as
inputs. In practice, one may want to use a much smaller value of $numtries$
for the confidence intervals than the one used in the \texttt{estnonreca}
procedure. Using the same number of $numtries$ would be time consuming; as
one would be making $ndraws\ast numtries$ maximization attempts.

To get Monte-Carlo-based confidence intervals pass an empty matrix [] for
the $residuals$.

\begin{eqnarray*}
\text{\lbrack cilmc,ciumc,cilvar,ciuvar,var]} &=&\text{%
mkimpci(betaz,a0,nlags,errshk,nstep, ...} \\
&&\text{ndraws,nobs,pctg,[],sigma,numtries,'mkmatrix','mkstart');}
\end{eqnarray*}

\begin{eqnarray*}
\text{\lbrack cilbt,ciubt,cilvar,ciuvar,var]} &=&\text{%
mkimpci(betaz,a0,nlags,errshk,nstep,...} \\
&&\text{ndraws,nobs,pctg,residuals,sigma,numtries,'mkmatrix','mkstart');}
\end{eqnarray*}

Note that the procedure calls should be one line, however, in order to fit
the page I put in a line break.

\subsection{Making Plots In Matlab with An Application to Confidence
Intervals}

Matlab graphics can be very basic and easy to make. To plot the first series
in data just type 
\begin{equation*}
\text{plot(data(:,1))}
\end{equation*}
A new figure window that contains the plot of the first column of data will
appear. This graph however is somewhat plain.

\begin{center}
\begin{tabular}{c}
A Plain Graph \\ 
\FRAME{itbpF}{3.0096in}{2.3947in}{0in}{}{}{plain.eps}{\special{language
"Scientific Word";type "GRAPHIC";maintain-aspect-ratio TRUE;display
"USEDEF";valid_file "F";width 3.0096in;height 2.3947in;depth
0in;original-width 6.6271in;original-height 5.2607in;cropleft "0";croptop
"1";cropright "1";cropbottom "0";filename 'plain.eps';file-properties
"NPEU";}}
\end{tabular}
\end{center}

I have included a series of commands below that will make a graph that looks
like this

\begin{center}
\begin{tabular}{c}
A More Interesting Graph \\ 
\FRAME{itbpF}{3.0536in}{2.4898in}{0in}{}{}{fig1.eps}{\special{language
"Scientific Word";type "GRAPHIC";maintain-aspect-ratio TRUE;display
"USEDEF";valid_file "F";width 3.0536in;height 2.4898in;depth
0in;original-width 6.7239in;original-height 5.4691in;cropleft "0";croptop
"1";cropright "1";cropbottom "0";filename 'fig1.eps';file-properties "NPEU";}%
}
\end{tabular}
\end{center}

This first step is to create a vector of series names. Using the Matlab
command \texttt{char, we} will create a series vector $varnames$ 
\begin{equation*}
\text{\texttt{varnames=char('Y','P','PCOM','FF','TOTR','NBR','M');}}
\end{equation*}

The next command \texttt{figure} creates a fresh graphics window. 
\begin{equation*}
\text{\texttt{figure}}
\end{equation*}

The next command \texttt{for} begins the loop 
\begin{equation*}
\text{\texttt{for zr = 1:4; }}
\end{equation*}

The next command \texttt{subplot} divides the page into pieces with 2 rows
and 2 columns. The third argument tells Matlab where to put the current
graph. 
\begin{equation*}
\text{\texttt{subplot(2,2,zr)}}
\end{equation*}

The next command \texttt{hold on} causes the \texttt{plot} command to put
one plot on top of another without erasing the other plots. 
\begin{equation*}
\text{\texttt{hold on}}
\end{equation*}

The next two lines plot the confidence intervals and the impulse response
for the $zr$ variable. The third line just sets the axis limits to the range
of the data. 
\begin{eqnarray*}
\text{\texttt{plot([100*cilvarb(}} &:&\text{\texttt{%
,zr)100*ciuvarb(:,zr)100*impzmat(:,zr)]);}} \\
\text{\texttt{plot([100*cilb(}} &:&\text{\texttt{,zr)100*ciub(:,zr)],'x-');}}
\\
&&\text{axis tight}
\end{eqnarray*}

The \texttt{title} command creates a title. The variable $varnames$ is used
to label the graph. 
\begin{equation*}
\text{\texttt{title(['Response by' varnames(zr,:)]);}}
\end{equation*}

The end command ends the loop. 
\begin{equation*}
\text{end;}
\end{equation*}

The final command suptitle is not a built in Matlab function but is included
with the VAR programs\footnote{%
The program \texttt{suptitle} was written by Drea Thomas of Mathworks.}. .
This function \texttt{suptitle} puts a title above the subplots. 
\begin{equation*}
\text{\texttt{suptitle('Shock to Fed Funds')}}
\end{equation*}

The Matlab manuals describes how to make graphs in much more detail.

\section{Variance Decomposition}

The variance decomposition is concerned with identifying the error in
forecasting a VAR $s$ periods into the future.

\begin{equation*}
Z_{t+s}-Z_{t+s|s}=u_{t+s}+\Psi _{1}u_{t+s-1}+\Psi _{2}u_{t+s-2}+...+\Psi
_{s-1}u_{t+1}
\end{equation*}
where $\Psi _{j}$ are the moving average coefficients and $Z_{t+s|s}$ is the
forecast at time $t$ of the value of $Z_{t+s}.$

The mean square error of the $s$-period forecast error is therefore

\begin{equation*}
\text{MSE}\left( Z_{t+s|s}\right) =V+\Psi _{1}V\Psi _{1}^{\prime }+\Psi
_{2}V\Psi _{2}^{\prime }+...+\Psi _{s-1}V\Psi _{s-1}^{\prime }
\end{equation*}

We are interested in knowing how much does each of the fundamental errors
contributes to this error term.

\begin{equation*}
Z_{t+s}-Z_{t+s|s}=A_{o}^{-1}\varepsilon _{t+s}+\Psi
_{1}A_{o}^{-1}\varepsilon _{t+s-1}+\Psi _{2}A_{o}^{-1}\varepsilon
_{t+s-2}+...+\Psi _{s-1}A_{o}^{-1}\varepsilon _{t+1}
\end{equation*}

Define $a_{j}$ to be the jth column of $A_{o}^{-1}$ then since $%
A_{o}^{-1}A_{o}^{-1\prime }$ equals V we can decompose the MSE as 
\begin{equation*}
\text{MSE}\left( Z_{t+s|s}\right) =\sum a_{j}a_{j}^{\prime }+\Psi
_{1}a_{j}a_{j}^{\prime }\Psi _{1}^{\prime }+\Psi _{2}a_{j}a_{j}^{\prime
}\Psi _{2}^{\prime }+...+\Psi _{s-1}a_{j}a_{j}^{\prime }\Psi _{s-1}^{\prime }
\end{equation*}
so the contribution of any one shock can be defined as 
\begin{equation*}
a_{j}a_{j}^{\prime }+\Psi _{1}a_{j}a_{j}^{\prime }\Psi _{1}^{\prime }+\Psi
_{2}a_{j}a_{j}^{\prime }\Psi _{2}^{\prime }+...+\Psi
_{s-1}a_{j}a_{j}^{\prime }\Psi _{s-1}^{\prime }
\end{equation*}

The percentage of the $s$-period ahead forecast error explained by a
particular shock can be easier to interpret\textbf{. \ }So for the
confidence intervals we will calculate.

\begin{equation*}
100\ast \frac{a_{j}a_{j}^{\prime }+\Psi _{1}a_{j}a_{j}^{\prime }\Psi
_{1}^{\prime }+\Psi _{2}a_{j}a_{j}^{\prime }\Psi _{2}^{\prime }+...+\Psi
_{s-1}a_{j}a_{j}^{\prime }\Psi _{s-1}^{\prime }}{\text{MSE}\left(
Z_{t+s|s}\right) }
\end{equation*}

\subsection{Matlab Code}

This code consists of two procedures. The first \texttt{vardecomp}
calculates the mean squared error and one error shock's contribution to the
mean squared error. The second \texttt{mkvdci } calculates a confidence
interval of the percentage contribution to the MSE.

The \texttt{vardecomp} procedure has the same inputs as the impulse response
function \texttt{mkimprep} has:. the VAR coefficients, the factorization of
the variance covariance matrix, the number of lags in the VAR, the choice of
error term, and the number of forecast period steps. The vardecomp procedure
returns two matrices. Both matrices $ms$ and msdec are a three dimensional
matrix. with dimensions $nvars$ by $nvars$ by $nstep$. The value $ms(j,k,s)$
is the $s$-period forecast covariance of the $j$-th and $k$-th elements of $%
Z_{t+s}.$ The value of $msdecomp(j,k,s)$ is how much the $errshk$-th
fundamental error contributes to the value of $ms(j,k,s)$. The procedure
call is

\begin{equation*}
\text{\lbrack ms,msdec]}=\text{vardecomp(betaz,a0,nlags,errshk,nstep);}
\end{equation*}

\subsubsection{Confidence Intervals}

Constructing confidence intervals for the variance decomposition is
identical to how they are constructed for the impulse responses. The only
difference is that the main procedure is \texttt{mkvdci} rather than \texttt{%
mkimpci} and the returns are different. Each gives the percentage of
variance explained by the $errshk$. The returns are three dimensional
matrices. with dimensions $nvars$ by $nvars$ by $nstep.$

\paragraph{Recursive}

Monte Carlo

\begin{equation*}
\text{\lbrack vdcilmc,vdciumc,vdcilvarmc,vdciuvarmc,vdvarmc]}=\text{%
mkvdci(betaz,a0,nlags,errshk,nstep,ndraws,nobs,pctg);}
\end{equation*}

Bootstrap

\begin{equation*}
\text{\lbrack
vdcilbt,vdciubt,vdcilvarbt,vdciuvarbt,vdvarbt]=mkvdci(betaz,a0,nlags,errshk,nstep,ndraws,nobs,pctg,residuals);%
}
\end{equation*}

\paragraph{Non-recursive}

Monte Carlo

\begin{eqnarray*}
\text{\lbrack vdcilmc,vdciumc,vdcilvarmc,vdciuvarmc,vdvarmc]} &=&\text{%
mkvdci(betaz,a0,nlags,errshk,nstep, ...} \\
&&\text{ndraws,nobs,pctg,[],sigma}{\tiny ,}\mathtt{numtries,\prime
mkmatrix\prime ,\prime mkstart\prime }\text{)};
\end{eqnarray*}

Bootstrap

\begin{eqnarray*}
\text{\lbrack vdcilbt,vdciubt,vdcilvarbt,vdciuvarbt,vdvarbt] } &=&\text{
mkvdci(betaz,a0,nlags,errshk,nstep, ...} \\
&&\text{ndraws,nobs,pctg,residuals,}\mathtt{sigma,numtries,\prime
mkmatrix\prime ,\prime mkstart\prime }\text{);}
\end{eqnarray*}

\section{Examples}

\subsection{Recursive example}

We want to estimate a model with 7 variables with 4 lags. The data has
already been prepared. We are interested in the impulse response to the
fourth shock. and want to have the 5 and 95\% confidence intervals. We also
want to create the first column of Table 3 which shows the Variance
Decomposition.
\begin{verbatim}
load datrep;
data=datrep;
%('Y','P','PCOM','FF','TOTR','NBR','M');
 
nlags  = 4;
hasconst = 1;
 
[betaz,sigma,residuals]=estimatevar(data,nlags,hasconst);
 
a0rec=inv(chol(sigma)');
 
errshk = 4;
nstep = 15;
 
impzmat=mkimprep(betaz,a0rec,nlags,errshk,nstep);
pctg=0.05;
ndraws=100;
nobs=120;
%[cilmc,ciumc,cilvar,ciuvar,var]=mkimpci(betaz,a0rec,nlags,errshk,nstep,ndraws,nobs,pctg);
[cilb,ciub,cilvarb,ciuvarb,varb]=mkimpci(betaz,a0rec,nlags,errshk,nstep,ndraws,...
nobs,pctg,residuals);
 
[mse,msedecomp]=vardecomp(betaz,a0rec,nlags,errshk,nstep);
 
vdpctg = msedecomp./mse;
 
[cilvd,ciuvd,cilvarbv,ciuvarbv,varbv]=mkvdci(betaz,a0rec,nlags,errshk,nstep,ndraws,nobs,pctg,residuals);
kstep = [2 4 8 12];
 
for zk=1:4;
   lowci(:,zk) = diag(squeeze(cilvd(:,:,kstep(zk))));
   lowvar(:,zk) = diag(squeeze(cilvarbv(:,:,kstep(zk))));
 
table(:,zk) = diag(squeeze(vdpctg(:,:,kstep(zk))));
highci(:,zk) = diag(squeeze(ciuvd(:,:,kstep(zk))));
highvar(:,zk) = diag(squeeze(ciuvarbv(:,:,kstep(zk))));
 
end;
 
%The first column of Table 3 from CEE
100*[lowci(:,1) table(:,1) highci(:,1)]
 
varnames=char('Y','P','PCOM','FF','TOTR','NBR','M');
 
figure
for zr=1:4;
   subplot(2,2,zr)
   hold on
   plot([ 100*cilvarb(:,zr) 100*ciuvarb(:,zr) 100*impzmat(:,zr)]);
   plot([100*cilb(:,zr) 100*ciub(:,zr)],'x-');
   title(['Response by ' varnames(zr,:)]);
   axis tight
end;
 
suptitle('Shock to Fed Funds')
figure
for zr=5:7;
   subplot(2,2,zr-4)
   hold on
      plot([ 100*cilvarb(:,zr) 100*ciuvarb(:,zr) 100*impzmat(:,zr)]);
   plot([100*cilb(:,zr) 100*ciub(:,zr)],'x-');
   title(['Response by ' varnames(zr,:)]);
   axis tight
end;
suptitle('Shock to Fed Funds')
 
\end{verbatim}

The result is the following impulse responses.

\begin{tabular}{cc}
\FRAME{itbpF}{2.7172in}{2.2157in}{0in}{}{}{fig1.eps}{\special{language
"Scientific Word";type "GRAPHIC";maintain-aspect-ratio TRUE;display
"USEDEF";valid_file "F";width 2.7172in;height 2.2157in;depth
0in;original-width 6.7239in;original-height 5.4691in;cropleft "0";croptop
"1";cropright "1";cropbottom "0";filename 'fig1.eps';file-properties "NPEU";}%
} & \FRAME{itbpF}{2.7172in}{2.2157in}{0in}{}{}{fig2.eps}{\special{language
"Scientific Word";type "GRAPHIC";maintain-aspect-ratio TRUE;display
"USEDEF";valid_file "F";width 2.7172in;height 2.2157in;depth
0in;original-width 6.7239in;original-height 5.4691in;cropleft "0";croptop
"1";cropright "1";cropbottom "0";filename 'fig2.eps';file-properties "NPEU";}%
}
\end{tabular}

{\Large Modification}. Change the ordering with the first variable being
interchanged with the fifth and the sixth variable being interchanged with
the seventh. This is very easy to do in Matlab. Replace the line \texttt{%
data=datrep; }with the line \texttt{data=datrep(:,[5 2 3 4 1 7 6];}. Nothing
else in the program needs to be modified. Of course, the results will be
different.

{\Large Modification}: Suppose we wanted to interchange the third and fourth
variables but were still interested in following the effects of the shock to
what is currently the fourth shock. First replace \texttt{data=datrep; }with
the line \texttt{data=datrep(:,[1:2 4 3 5:7]);}. Second the line \texttt{%
errshk = 4}; needs to be changed to \texttt{errshk = 3;}. Again that is all.

{\Large Modification }Only using a subperiod of the data. Suppose instead of
using the 1+nlags to nobs observations you only wanted to use a subset of
the data from say abeg to aend. As long as abeg is greater than or equal to
1+ $nlags$ and aend is less than or equal to $nobs$ then this modification
can be again made at the data statement. \ Insert the line \texttt{%
data=datrep(abeg-nlags:aend,:);}

{\Large Modification }Only using a subset of the variables. want to exclude
the 2nd and the 6th variables Set \texttt{data=datrep(:,[1 3 4 5 7]); Change
errshk = 3.}

\subsubsection{Nonrecursive example}

We now estimate the Sims Zha model described in CEE. Keep in mind that
nonrecursive models take much much longer to estimate than recursive models.
This program assumes that you have created the two files \texttt{mksimzha}
and \texttt{svsz} to create the $A_{0}$ matrix and the starting values. A
listingof these two files is below the current file.

{\Large nonrecmain.m}

\texttt{load c:\TEXTsymbol{\backslash}robswork\TEXTsymbol{\backslash}%
identeich\TEXTsymbol{\backslash}szdat\_dm.dat;}

\texttt{data=szdat\_dm;}

\texttt{data(:,1)=data(:,1)./sqrt(10.0); }

\texttt{nl = 4;}

\texttt{[bz,sig,rz]=estimatevar(data,nl);}

\texttt{a0=estnonreca(sig,20,'mksimzha','svsz')}

\texttt{eshk=3;}

\texttt{nstp = 15;}

\texttt{\ }

\texttt{\ }

\texttt{\ impz=mkimprep(bz,a0,nl,eshk,nstp);}

\texttt{pc=0.05;}

\texttt{nd=10; \%should be more like a 100}

\texttt{nb=120;}

\texttt{%
[lp,up,lv,uv]=mkimpci2(bz,a0,nl,eshk,nstp,nd,nb,pc,rz,sig,3,'mksimzha','svsz');%
}

\texttt{[mse,msed]=vardecomp(bz,a0,nl,eshk,nstp);}

\texttt{vdpctg = msed./mse;}

\texttt{%
[vlp,vup,vlv,vuv]=mkvdci(bz,a0,nl,eshk,nstp,nd,nb,pc,rz,sig,3,'mksimzha','svsz');%
}

\texttt{kstep = [2 4 8 12];}

\texttt{for zk=1:4;}

\texttt{\ lowci(:,zk) = diag(squeeze(vlp(:,:,kstep(zk))));}

\texttt{\ lowvar(:,zk) = diag(squeeze(vlv(:,:,kstep(zk))));}

\texttt{table(:,zk) = diag(squeeze(vdpctg(:,:,kstep(zk))));}

\texttt{highci(:,zk) = diag(squeeze(vup(:,:,kstep(zk))));}

\texttt{highvar(:,zk) = diag(squeeze(vuv(:,:,kstep(zk))));}

\texttt{end;}

\texttt{\ }

\texttt{\ }

\texttt{varnames=char('Y','P','PCOM','FF','TOTR','NBR','M');}

\texttt{for zr=1:7;}

\texttt{\ subplot(4,2,zr)}

\texttt{\ hold on}

\texttt{\ plot([impz(:,zr) cilb(:,zr) ciub(:,zr)],'.');}

\texttt{\ \% title(['Response by ' varnames(zr,:)]); }

\texttt{\ axis tight}

\texttt{end;}

\bigskip

\bigskip

{\Large mksimzha.m}

function [azero]=mksimzha(x);

x=x';

azero=[x(1:7); 0 x(8:9) -x(8) 0 -x(8) 0; x(10:12)zeros(1,4); x(13)
zeros(1,2) x(14:17);...

x(18) zeros(1,3) x(19:21) ;x(22) zeros(1,4) x(23:24);x(25) zeros(1,5) x(26)];

\bigskip

{\Large svsz.m}

function avec=svsz;

avec=2*randn(26,1);

avec([1 8 12 14 19 23 26]) = abs(avec([1 8 12 14 19 23 26]));

\end{document}
