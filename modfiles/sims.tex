\documentclass[a4paper,11pt]{article}
\title{RBC1}
\author{Eric Scheffel}
\begin{document}

\maketitle


The problem is set up as a social planner's problem in the following way:

\begin{eqnarray*}
L & = & E_0\sum_{t=0}^\infty\beta^t\left\lbrace\frac{c_t^{\left(1-\eta\right)}}{1-\eta}\right\rbrace\\
  & + & \lambda_t\left[{z_t}{k_t}^\rho-c_t+\left(1-\delta\right)k_t-k_{t+1}\right]
\end{eqnarray*}
The shock process is given by:
\[z_{t+1}=\bar{z}^{1-\psi}z_t^\psi\epsilon_{t+1}\]
Market equilibrium requires:
\[{z_t}{k_t}^\rho = c_t+k_{t+1}-\left(1-\delta\right)k_t\]
which is just a repetition of the budget constraint.
The first order conditions to this problem are:
\[
c_t^{-\eta}=\lambda_t;
\]
\[
\beta E_t\lambda_{t+1}\left[z_{t+1}\rho k_{t+1}^{\rho-1}+\left(1-\delta\right)\right]=\lambda_t;
\]
But using the following definition for the gross real interest rate:
\[
R_t=z_t\rho k_t^{\rho-1}+\left(1-\delta\right)
\]
and recalling market equilibrium and the error process, the whole system is equal to:
\begin{equation}
{z_t}{k_t}^\rho-c_t-k_{t+1}+\left(1-\delta\right)k_t=0;
\end{equation}
which is market equilibrium
\begin{equation}
z_t\rho k_t^{\rho-1}+\left(1-\delta\right)-R_t=0;
\end{equation}
which is the definition of the gross real interest rate
\begin{equation}
c_t^\eta\beta E_t c_{t+1}^{-\eta}R_{t+1}-1=0;
\end{equation}
which is the consumption Euler equation
\begin{equation}
{z_t}{k_t}^\rho-y_t=0;
\end{equation}
which is the definition for output
\begin{equation}
\bar{z}^{1-\psi}z_t^\psi\epsilon_{t+1}-z_{t+1}=0;
\end{equation}
which is the shock process. Notice that the steady state is numerically
calculated according to the following system
\[\bar{z}\bar{k}^\rho-\bar{c}-\delta \bar{k}=0;\]
\[\bar{z}\rho\bar{k}^{\rho-1}+\left(1-\delta\right)-\bar{R}=0;\]
\[\beta\bar{R}-1=0;\]
\[\bar{z}\bar{k}^\rho-\bar{y}=0;\]
\end{document}